
\documentclass[12pt]{article}

\usepackage{geometry}
\geometry{margin=1in}

\title{TestRail Snapshot 2 Report}
\date{}
\begin{document}
\maketitle

\section*{Overview}
This document summarizes the TestRail test run associated with Snapshot 2 of the Career Help Platform. 
The primary focus of this test run is the AI Career Roadmap Generator feature, which allows students to request 
a personalized multi-phase roadmap for reaching their target careers.

\section*{Test Run Details}
\begin{itemize}
    \item \textbf{Feature Under Test:} AI Career Roadmap Generator
    \item \textbf{Scope:} Basic functionality, input handling, error handling, and response formatting.
    \item \textbf{Tester(s):} Project team members.
\end{itemize}

\section*{Test Cases}

\subsection*{TC-CR1: Roadmap Request with Complete Information}
\begin{itemize}
    \item \textbf{Precondition:} The TestRail run is configured and the application is accessible.
    \item \textbf{Steps:}
    \begin{enumerate}
        \item Log into the Career Help Platform.
        \item Open the chatbot interface.
        \item Enter a request such as ``Create a roadmap for a Computer Science student who wants to become a software engineer.''
    \end{enumerate}
    \item \textbf{Expected Result:} A clear, multi-phase roadmap is generated and displayed to the user.
\end{itemize}

\subsection*{TC-CR2: Handling Missing Inputs}
\begin{itemize}
    \item \textbf{Steps:}
    \begin{enumerate}
        \item Open the chatbot interface.
        \item Enter a request such as ``Help me plan my career'' without specifying a major.
    \end{enumerate}
    \item \textbf{Expected Result:} The chatbot prompts the user to provide their major and any other required information before generating a roadmap.
\end{itemize}

\subsection*{TC-CR3: Invalid Input Text}
\begin{itemize}
    \item \textbf{Steps:}
    \begin{enumerate}
        \item Enter nonsensical or random characters as the request.
    \end{enumerate}
    \item \textbf{Expected Result:} The chatbot indicates that it could not understand the request and asks the user to clarify their goal.
\end{itemize}

\subsection*{TC-CR4: AI Service Timeout or Failure}
\begin{itemize}
    \item \textbf{Steps:}
    \begin{enumerate}
        \item Trigger a roadmap request while the AI backend is unavailable (simulated in a test environment if necessary).
    \end{enumerate}
    \item \textbf{Expected Result:} The system returns a friendly error message explaining that the roadmap service is temporarily unavailable and suggests trying again later.
\end{itemize}

\subsection*{TC-CR5: Response Formatting Quality}
\begin{itemize}
    \item \textbf{Steps:}
    \begin{enumerate}
        \item Submit a standard roadmap request.
    \end{enumerate}
    \item \textbf{Expected Result:} The roadmap is organized into headings or bullet points (for example, ``Year 1'', ``Year 2'') so that guidance is easy to read.
\end{itemize}

\section*{Conclusion}
The Snapshot 2 TestRail run provides baseline coverage for the new AI Career Roadmap Generator feature. 
Additional test cases and more detailed execution results can be added in future snapshots as the implementation is refined.

\end{document}
