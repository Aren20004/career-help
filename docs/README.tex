\documentclass[12pt]{article}

\usepackage{geometry}
\geometry{margin=1in}

\title{Career Help Platform User Manual}
\date{}

\begin{document}
\maketitle

\section{Overview}
The Career Help Platform is designed to guide users through their professional development by providing personalized recommendations for colleges, certificates, internships, job opportunities, and skill-building experiences. 
The system allows users to select a career path and receive a curated roadmap tailored to their goals.

\section{Purpose of the Software}
The goal of this platform is to assist students and job seekers who are unsure of the steps needed to advance in their chosen career. 
By centralizing key resources, the platform reduces confusion and helps users plan their journey effectively.

\section{How to Access the System}
The system is currently in a documentation and planning phase.  
Future versions will include a fully accessible web interface.

\section{Jira Link}
All project tasks, sprints, and planning documents are managed in Jira.  
Insert your Jira link here once generated.

\section{Objectives}
\begin{itemize}
    \item Provide personalized career roadmaps.
    \item Centralize information about education, experience, and job requirements.
    \item Support long-term professional planning.
    \item Future feature: Paid 1-on-1 counseling with human advisors.
\end{itemize}


\section{Snapshot 2 Feature: AI Career Roadmap Generator}

The Snapshot 2 update introduces an AI-driven career roadmap generator designed specifically for students. 
Instead of only selecting a static career path, students can now describe their situation and goals in natural language 
and receive a structured set of recommendations.

\subsection{How Students Use the Feature}
\begin{itemize}
    \item Open the chatbot interface in the Career Help Platform.
    \item Type a request such as ``Create a career roadmap for a first-year Computer Science student who wants to become a backend developer''.
    \item Answer any follow-up questions about interests, skills, or timeline.
    \item Review the generated roadmap, which is broken into phases (for example, Year 1--4).
\end{itemize}

\subsection{What the Roadmap Includes}
\begin{itemize}
    \item Recommended types of classes or topics to prioritize.
    \item Suggestions for clubs, hackathons, or side projects.
    \item Guidance on when to search for internships and entry-level roles.
    \item Ideas for portfolio building and resume improvement.
\end{itemize}

This feature lays the groundwork for future enhancements, such as directly integrating with class schedule data and external internship listings.



\section{Snapshot 3 Feature: Internship Opportunities and Guidance}

Students may now:
\begin{itemize}
\item Request internship lists (e.g., ``Find CS internships in LA'').
\item Ask how to get internships (resume, timeline, networking advice).
\item View internship examples returned by the chatbot.
\end{itemize}

\end{document}
\section*{Snapshot 4 Final Update}

During Snapshot 4, the Career Help Platform was finalized and prepared for release. 
This included UI cleanup, documentation completion, workflow updates, and final system testing.

\subsection*{Completed for Snapshot 4}
\begin{itemize}
    \item Finalized SDD, SRS, Design Specification, and Snapshot Objectives documents.
    \item Added TestRail Snapshot 4 test summary.
    \item Updated workflow diagram.
    \item Performed front-end and back-end cleanup for consistency.
    \item Added Docker configuration for deployment.
\end{itemize}

\section*{How to Run the Platform}

\subsection*{Backend}
\begin{verbatim}
cd backend
npm install
npm start
\end{verbatim}

\subsection*{Frontend}
\begin{verbatim}
cd frontend
npm install
npm run dev
\end{verbatim}

\section*{Jira Board Link}
\texttt{<insert-your-jira-board-link-here>}

\section*{Dependencies Added in Snapshot 4}
No new dependencies were required for Snapshot 4; the focus was stability and documentation.
