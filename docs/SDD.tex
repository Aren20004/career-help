\documentclass[12pt]{article}

\usepackage{geometry}
\geometry{margin=1in}
\usepackage{longtable}

\begin{document}

\begin{titlepage}
\centering
\vspace*{4cm}
{\Huge \textbf{Software Design Document}}\\[1cm]
{\Large Career Help Platform}\\[0.5cm]
{\large Version 1.0}\\[2cm]
\vfill
\end{titlepage}

\tableofcontents
\newpage

\section*{Version Description}
\addcontentsline{toc}{section}{Version Description}

\begin{longtable}{|c|c|c|}
\hline
Version & Description & Date \\ \hline
1.0 & Initial SDD for Snapshot 1 & 12-08-2025 \\ \hline
\end{longtable}

\section{Introduction}
The SDD outlines the system architecture and design for the Career Help Platform.

\subsection{Purpose}
To guide future developers and testers in understanding the system structure.

\subsection{Intended Audience}
Developers, instructors, testers, and stakeholders.

\section{System Architecture}
The architecture consists of:
\begin{itemize}
    \item A frontend user interface.
    \item A backend recommendation engine.
    \item A structured data layer storing career resources.
\end{itemize}

Workflow:
\begin{enumerate}
    \item User selects a career.
    \item Backend retrieves associated education, certificates, internships, jobs.
    \item Recommendations appear on the UI.
\end{enumerate}

\section{User Interface}
\begin{itemize}
    \item Simple layout with career selection and results panel.
    \item User-friendly and accessible.
\end{itemize}

\section{Glossary}
\begin{longtable}{|c|p{10cm}|}
\hline
Term & Definition \\ \hline
Career Path & A structured journey toward a target profession. \\ \hline
Recommendation Engine & Component that returns resources associated with a selected career. \\ \hline
\end{longtable}


\section{Snapshot 2 Design Additions}

\subsection{AI Career Roadmap Generator Component}
The Snapshot 2 milestone adds the AI Career Roadmap Generator as a core backend component of the Career Help Platform. 
This component is responsible for turning structured student input into a multi-phase roadmap that outlines recommended classes, 
skills, projects, and internship milestones.

\subsubsection{Responsibilities}
\begin{itemize}
    \item Accept student information such as major, interests, experience, and target career.
    \item Construct prompts or structured requests for the AI model or rule-based engine.
    \item Post-process AI output into a structured format (for example, phases or years).
    \item Return the roadmap to the chatbot interface for display to the student.
\end{itemize}

\subsubsection{Interfaces}
\begin{itemize}
    \item Exposes a backend endpoint (for example, \texttt{/career-roadmap}) that receives JSON input.
    \item Communicates with an AI provider or local model using an SDK or HTTP API.
    \item Returns a JSON structure that can be rendered by the frontend chat interface.
\end{itemize}

\subsection{Updated Architecture Overview}
The high-level architecture is updated to include the AI Career Roadmap Generator as part of the backend recommendation engine. 

\begin{itemize}
    \item \textbf{Chat UI / Frontend}: Provides a text-based interface where students can ask questions and request a career roadmap.
    \item \textbf{Chatbot Orchestrator}: Parses user intent (for example, ``create a career roadmap for a CS major''), validates inputs, and routes the request.
    \item \textbf{Career Roadmap Service}: New module that builds prompts, calls the AI model, and formats route responses.
    \item \textbf{Data Layer}: Optionally enriches responses with static data (for example, predefined skills or example courses).
\end{itemize}

\subsection{Updated Workflow}
The workflow is expanded to show how a roadmap request flows through the system:

\begin{enumerate}
    \item The student opens the chatbot interface and types a request such as ``Create a roadmap for a Computer Science major who wants to become a software engineer.''
    \item The frontend sends the request to the backend along with any structured form inputs (major, interests, experience, goal).
    \item The backend chatbot orchestrator analyzes the request and calls the Career Roadmap Service.
    \item The Career Roadmap Service generates a prompt and invokes the AI model or rule-based logic.
    \item The AI response is parsed into sections (for example, Year 1--4 or phases) and wrapped in a structured response object.
    \item The formatted roadmap is returned to the frontend and displayed to the student inside the chat.
\end{enumerate}

\subsection{New Dependencies}
To support this feature, the design assumes the following additional dependencies will be included in the implementation:

\begin{itemize}
    \item AI model SDK or HTTP client library to communicate with a large language model.
    \item Configuration management (for example, environment variables) for securely storing API keys.
\end{itemize}



\section{Snapshot 3 Design Additions}

\subsection{Internship Opportunities and Guidance Module}
This module enables internship discovery and tailored application support.

\subsubsection{Responsibilities}
\begin{itemize}
\item Accept student internship preferences.
\item Match criteria to internship opportunities (mock data or future API).
\item Produce application guidance (resume, networking, timelines).
\end{itemize}

\subsection{Updated Architecture Overview}
\begin{itemize}
\item Chat UI extended to support internship queries.
\item Intent router directs internship queries to new module.
\item Internship Service fetches matches and generates guidance.
\end{itemize}

\subsection{Workflow Extension}
\begin{enumerate}
\item Student requests internship info.
\item Query routed to internship module.
\item Module returns internship results + guidance.
\end{enumerate}

\end{document}
