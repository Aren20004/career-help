\documentclass[11pt]{article}
\usepackage[margin=1in]{geometry}
\usepackage{hyperref}
\usepackage{booktabs}
\usepackage{longtable}
\usepackage{array}
\usepackage{enumitem}

\hypersetup{
    colorlinks=true,
    linkcolor=black,
    urlcolor=blue
}

\begin{document}

\begin{center}
    {\LARGE \textbf{Career Help Project -- Jira Overview}}\\[0.5em]
    {\large Bug Tracker (CH) \& Scrum Board (CHS)}\\[1em]
    \today
\end{center}

\section{Jira Project Links}

\subsection{Bug Tracker (Project CH)}
This Jira view is used as the main bug tracker for the project.

\begin{itemize}[leftmargin=*]
    \item \textbf{Project key:} CH
    \item \textbf{Description:} Central place to report, track, and prioritize bugs.
    \item \textbf{Jira link:}\\
    \url{https://cs3338-career-help-project.atlassian.net/jira/software/c/projects/CH/list?jql=project%20%3D%20%22CH%22%20ORDER%20BY%20created%20DES}
\end{itemize}

\subsection{Scrum Board (Project CHS)}
This Jira board is used to track work in sprints for the Career Help system.

\begin{itemize}[leftmargin=*]
    \item \textbf{Project key:} CHS
    \item \textbf{Description:} Scrum board for tracking sprint work, including stories, tasks, and bugs.
    \item \textbf{Jira board link:}\\
    \url{https://cs3338-career-help-project.atlassian.net/jira/software/projects/CHS/boards/166?sprintStarted=true}
\end{itemize}

\section{Bug Tracker Structure (CH)}

This section describes how bugs in the \textbf{CH} project are typically organized. You can adjust this text and the tables below to match your actual workflow.

\subsection{Recommended Bug Fields}

\begin{itemize}[leftmargin=*]
    \item \textbf{Issue Key}: Jira-generated ID (e.g., CH-101)
    \item \textbf{Summary}: Short description of the bug
    \item \textbf{Status}: e.g., Open, In Progress, In Review, Done
    \item \textbf{Priority}: e.g., Critical, High, Medium, Low
    \item \textbf{Assignee}: Person responsible for fixing the bug
    \item \textbf{Reporter}: Person who reported the bug
    \item \textbf{Created / Updated}: Timestamps for lifecycle tracking
    \item \textbf{Sprint / Fix Version}: If the bug is tied to a sprint or release
\end{itemize}

\subsection{Sample Bug List (to be filled in manually)}

\setlength{\extrarowheight}{2pt}
\begin{longtable}{>{\raggedright\arraybackslash}p{1.8cm}
                  >{\raggedright\arraybackslash}p{5.2cm}
                  >{\raggedright\arraybackslash}p{1.7cm}
                  >{\raggedright\arraybackslash}p{1.7cm}
                  >{\raggedright\arraybackslash}p{2cm}
                  >{\raggedright\arraybackslash}p{2.4cm}}
\toprule
\textbf{Issue Key} & \textbf{Summary} & \textbf{Status} & \textbf{Priority} & \textbf{Assignee} & \textbf{Created (Date)} \\
\midrule
\endfirsthead

\toprule
\textbf{Issue Key} & \textbf{Summary} & \textbf{Status} & \textbf{Priority} & \textbf{Assignee} & \textbf{Created (Date)} \\
\midrule
\endhead

CH-001 & Example bug: Login button unresponsive on homepage & Open & High & Unassigned & 2025-01-01 \\
CH-002 & Example bug: Error message not shown for invalid password & In Progress & Medium & Student A & 2025-01-02 \\
CH-003 & Example bug: Profile picture upload fails on mobile & In Review & High & Student B & 2025-01-03 \\
CH-004 & Example bug: Resume download link broken & Done & Critical & Student C & 2025-01-04 \\

\bottomrule
\end{longtable}

\section{Scrum Board Structure (CHS)}

This section describes how the \textbf{CHS} scrum board might be structured around sprints.

\subsection{Typical Columns}

\begin{itemize}[leftmargin=*]
    \item \textbf{Backlog}
    \item \textbf{To Do}
    \item \textbf{In Progress}
    \item \textbf{In Review / Testing}
    \item \textbf{Done}
\end{itemize}

\subsection{Sprint Overview Template}

You can duplicate the structure below for each sprint in the CHS board.

\subsubsection*{Sprint Name: \textit{Sprint 1 -- Example}}
\begin{itemize}[leftmargin=*]
    \item \textbf{Sprint Goal:} Describe the main objective of the sprint.
    \item \textbf{Start Date:} YYYY-MM-DD
    \item \textbf{End Date:} YYYY-MM-DD
    \item \textbf{Total Issues:} X
    \item \textbf{Completed Issues:} Y
    \item \textbf{Velocity (Story Points):} Z
\end{itemize}

\subsection{Sample Sprint Issue Table}

\begin{longtable}{>{\raggedright\arraybackslash}p{1.8cm}
                  >{\raggedright\arraybackslash}p{5.2cm}
                  >{\raggedright\arraybackslash}p{2cm}
                  >{\raggedright\arraybackslash}p{1.8cm}
                  >{\raggedright\arraybackslash}p{2cm}}
\toprule
\textbf{Issue Key} & \textbf{Summary} & \textbf{Issue Type} & \textbf{Status} & \textbf{Story Points} \\
\midrule
\endfirsthead

\toprule
\textbf{Issue Key} & \textbf{Summary} & \textbf{Issue Type} & \textbf{Status} & \textbf{Story Points} \\
\midrule
\endhead

CHS-001 & Example story: As a user, I can create an account & Story & Done & 3 \\
CHS-002 & Example task: Implement email verification & Task & In Progress & 5 \\
CHS-003 & Example bug: Verification email not sent & Bug & To Do & 2 \\
CHS-004 & Example story: As a user, I can upload my resume & Story & Done & 5 \\

\bottomrule
\end{longtable}

\section{Notes and Next Steps}

\begin{itemize}[leftmargin=*]
    \item Replace the example issues with real data from Jira (export as CSV or copy-paste).
    \item You can duplicate the bug and sprint tables for multiple releases/sprints.
    \item Consider adding sections for ``Retrospective'', ``Risks'', and ``Action Items'' if this document is used as a formal report.
\end{itemize}

\end{document}
