\documentclass[12pt]{article}

\usepackage{geometry}
\geometry{margin=1in}
\usepackage{longtable}

\begin{document}

\begin{titlepage}
\centering
\vspace*{4cm}
{\Huge \textbf{Software Requirements Specification}}\\[1cm]
{\Large Career Help Platform}\\[0.5cm]
{\large Version 1.0}\\[2cm]
\vfill
\end{titlepage}

\tableofcontents
\newpage

\section*{Version Description}
\addcontentsline{toc}{section}{Version Description}

\begin{longtable}{|c|c|c|}
\hline
\textbf{Version} & \textbf{Description} & \textbf{Date} \\ \hline
1.0 & Initial version for Snapshot 1 & 12-08-2025 \\ \hline
\end{longtable}

\section{Introduction}
\subsection{Purpose}
This Software Requirements Specification (SRS) defines the functional and non-functional requirements for the Career Help Platform. The system provides users with career pathways including educational resources, certifications, internships, and job recommendations.

\subsection{Intended Audience}
\begin{itemize}
    \item Developers planning future implementation.
    \item Testers evaluating system requirements.
    \item Students, advisors, and project stakeholders.
\end{itemize}

\subsection{Overview}
The system will guide users through selecting a career and receiving a personalized roadmap. Future versions will include optional paid counseling.

\section{External Interface Requirements}
\subsection{User Interface}
\begin{itemize}
    \item Career selection dropdown.
    \item Results section displaying required education, certifications, experience, and jobs.
    \item Clean, web-based UI.
\end{itemize}

\subsection{Software Interface}
\begin{itemize}
    \item Backend recommendation engine (planned).
    \item Data storage using structured JSON or database.
\end{itemize}

\section{Legal and Ethical Considerations}
\begin{itemize}
    \item User preferences are not to be stored without consent.
    \item Recommendations must avoid bias or discrimination.
    \item Paid counseling information must be disclosed transparently.
\end{itemize}


\section{Snapshot 2 Functional Requirements}

\subsection{New Functional Requirements for Career Roadmap Generation}

\begin{itemize}
    \item \textbf{FR-CR1: Collect Student Inputs} \\
    The system shall allow a student to provide their major or program, areas of interest, current academic standing, and target career goal.

    \item \textbf{FR-CR2: Generate Multi-Phase Roadmap} \\
    The system shall generate a multi-phase roadmap that includes suggested academic steps, skills to build, project ideas, and internship milestones relevant to the student\'s goal.

    \item \textbf{FR-CR3: Chat-Based Interaction} \\
    The system shall allow the student to request a career roadmap through a chat-style interface using natural language queries.

    \item \textbf{FR-CR4: Output Formatting} \\
    The roadmap shall be returned in a readable format, broken into clear phases such as Year 1--4 or equivalent terms, and presented in the chatbot output.

    \item \textbf{FR-CR5: Handling Missing Information} \\
    If the student does not provide enough information (for example, no major or goal), the system shall ask follow-up questions before generating a roadmap.
\end{itemize}

\subsection{Updated Non-Functional Requirements}

\begin{itemize}
    \item \textbf{NFR-CR1: Response Time} \\
    Under normal load, a roadmap request should complete in 10 seconds or less, assuming the external AI service is available.

    \item \textbf{NFR-CR2: Clarity of Guidance} \\
    The roadmap must be presented in a clear, organized way that is understandable to students with no prior experience in career planning.

    \item \textbf{NFR-CR3: Privacy and Ethics} \\
    Student inputs used to generate the roadmap must not be stored permanently without consent and should not include sensitive personal identifiers unless absolutely necessary.
\end{itemize}



\section{Snapshot 3 Functional Requirements}

\subsection{Internship Opportunities and Guidance Requirements}
\begin{itemize}
\item FR-INT1: System accepts internship search criteria.
\item FR-INT2: System retrieves or generates internships.
\item FR-INT3: Natural language internship queries supported.
\item FR-INT4: System provides application guidance.
\item FR-INT5: Align internships with roadmap.
\end{itemize}

\subsection{Snapshot 3 Non-Functional Requirements}
\begin{itemize}
\item NFR-INT1: Clear and student-friendly internship descriptions.
\item NFR-INT2: Response time under 10 seconds.
\item NFR-INT3: Sensitive user info handled responsibly.
\end{itemize}

\end{document}

% -----------------------------------------------------
\section{Snapshot 4 Final Requirements}
% -----------------------------------------------------

\subsection{Functional Requirements}
Snapshot 4 does not introduce new major features. Instead, it finalizes and stabilizes the platform.

\begin{itemize}
    \item \textbf{FR-S4-1: System Cleanup}\\
    The system shall remove unused components, fix UI spacing, reduce redundancy, and ensure consistent navigation.

    \item \textbf{FR-S4-2: Documentation Completion}\\
    The system shall include updated versions of the SRS, SDD, Design Specification, User Manual, and workflow diagram.

    \item \textbf{FR-S4-3: TestRail Validation}\\
    All previously implemented features shall be validated through Snapshot 4 test cases, with results documented.

    \item \textbf{FR-S4-4: Deployment Support}\\
    The system shall include a functional docker-compose file capable of running all major services (frontend, backend, database).
\end{itemize}

\subsection{Non-Functional Requirements}
\begin{itemize}
    \item \textbf{NFR-S4-1: Stability}\\
    The final system shall be stable enough for demonstration purposes, with no critical errors during navigation.

    \item \textbf{NFR-S4-2: Documentation Quality}\\
    All documentation must be complete, readable, and aligned across Snapshot versions.

    \item \textbf{NFR-S4-3: Workflow Accuracy}\\
    The final workflow diagram must correctly represent the system’s architecture and data flow.

    \item \textbf{NFR-S4-4: Test Coverage}\\
    All major features from Snapshots 2 and 3 must be tested in TestRail.
\end{itemize}

% -----------------------------------------------------
\section{Future Work}
% -----------------------------------------------------

Future expansions of the Career Help Platform may include:
\begin{itemize}
    \item Role-based accounts for students, advisors, and administrators.
    \item Real-time job and internship data via external APIs.
    \item Personalized dashboards tracking student progress.
    \item Integration of salary projections and industry trends.
    \item Mobile app version (iOS and Android).
    \item Notification system for application deadlines and new opportunities.
\end{itemize}

\end{document}
