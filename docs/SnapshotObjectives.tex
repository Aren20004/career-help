\documentclass[12pt]{article}

\usepackage{geometry}
\geometry{margin=1in}

\title{Snapshot Objectives}
\date{}
\begin{document}
\maketitle

\section*{Snapshot 1 — Start Objectives}

The purpose of Snapshot 1 is to establish the foundation of the Career Help Platform. 
This phase focuses entirely on planning, project structure, and documentation requirements. 
The system architecture, goals, and initial design decisions are defined here.

\begin{itemize}
    \item \textbf{Define the project scope:}  
    The Career Help Platform assists users in selecting career paths and provides curated recommendations for colleges, certificates, internships, jobs, and experience-building opportunities.

    \item \textbf{Establish core architecture:}  
    A web-based frontend, backend recommendation engine, and a structured data layer are planned.  
    A future paid counseling module will be integrated in later snapshots.

    \item \textbf{Create initial documentation:}  
    Version 1.0 of the SRS, SDD, README/User Manual, and Design Specification are created.

    \item \textbf{Plan project dependencies and technologies:}  
    The system design includes Python Flask for backend planning and HTML/CSS/JS for the frontend (no code implementation required at this stage).  
    Docker containerization will be incorporated in future snapshots.

    \item \textbf{Create Jira board:}  
    Tasks are created for UI layout, backend logic planning, data design, documentation work, and future bug tracking.

    \item \textbf{Create workflow diagram:}  
    A high-level depiction showing user input flowing through the UI, backend recommendation logic, data retrieval, and final results displayed to the user.
\end{itemize}


\section*{Snapshot 2 --- First Checkpoint Objectives}

\subsection*{Reflection on Snapshot 1}
Snapshot 1 focused on setting up the foundation of the Career Help Platform. 
The team created the initial versions of the SRS, SDD, User Manual, and Design Specification, 
and defined the overall problem space: helping students explore careers and understand the paths required to reach them. 
At this stage, the chatbot, AI logic, and dynamic roadmap generation were still in the planning phase.

\subsection*{New Objective: AI-Powered Career Roadmap Generator for Students}
Snapshot 2 introduces the first major functional feature of the system: an AI-powered career roadmap generator 
targeted specifically at students. The feature will provide structured multi-step guidance to help students move from 
their current academic status toward their desired career.

\subsection*{Goals for Snapshot 2}
\begin{itemize}
    \item Implement a new backend service that accepts student inputs such as major, interests, experience level, and target career goal.
    \item Integrate an AI model or external API capable of generating a multi-year roadmap with clear milestones.
    \item Extend the chatbot interface so students can request a ``career roadmap'' using natural language.
    \item Include recommendations for relevant classes, internships, projects, and skills that support the student\'s target career.
    \item Ensure responses are formatted in phases (for example, Year 1--4 or Semester 1--8) to help students plan over time.
    \item Lay the groundwork for future extensions such as class schedule lookup and internship search integration.
\end{itemize}

\subsection*{Deliverables for Snapshot 2}
\begin{itemize}
    \item Updated SRS, SDD, User Manual, and Design Specification including the career roadmap generator feature.
    \item Updated workflow diagram showing the new AI roadmap component integrated with the chatbot.
    \item Backend endpoint stubbed or implemented for generating roadmaps.
    \item Basic test cases documented in TestRail for roadmap requests, validation, and error handling.
\end{itemize}



\section*{Jira Tasks for Snapshot 2}

For Snapshot 2, the Jira board is updated with an epic and related stories focused on the AI Career Roadmap Generator. 
Below is a high-level summary of the planned work items that appear in Jira.

\subsection*{Epic: AI Career Roadmap Generator}
\textbf{Goal:} Enable students to request a personalized multi-phase career roadmap through the chatbot interface.

\subsection*{User Stories and Tasks}
\begin{itemize}
    \item \textbf{Story 1: Collect Student Career Input}
    \begin{itemize}
        \item Define the input fields for roadmap generation (major, interests, experience level, career goal).
        \item Update the frontend to prompt the user for missing information.
        \item Validate request data on the backend.
        \item Document the input schema in the API specification.
    \end{itemize}

    \item \textbf{Story 2: Develop Roadmap Generation Engine}
    \begin{itemize}
        \item Create a prompt template for generating roadmaps with an AI model.
        \item Implement a placeholder function or service to call the AI or rule-based engine.
        \item Parse and format the AI output into phases (for example, Year 1--4).
        \item Add basic tests for the roadmap formatting logic.
    \end{itemize}

    \item \textbf{Story 3: Chatbot Integration}
    \begin{itemize}
        \item Add a ``roadmap'' intent to the chatbot workflow.
        \item Ensure the chatbot can ask follow-up questions when information is missing.
        \item Display roadmap results as clearly separated sections in the chat UI.
    \end{itemize}

    \item \textbf{Story 4: Documentation Updates}
    \begin{itemize}
        \item Update the SDD with the Career Roadmap Service component.
        \item Update the SRS with new functional and non-functional requirements.
        \item Update the User Manual and Design Specification to describe how students use the feature.
    \end{itemize}

    \item \textbf{Story 5: Workflow Diagram Update}
    \begin{itemize}
        \item Extend the high-level workflow diagram to include the roadmap request, AI processing, and response.
    \end{itemize}
\end{itemize}

\subsection*{Potential Bug Tickets}
The following potential bugs are also created in Jira for future testing and tracking:
\begin{itemize}
    \item BUG-01: Chatbot does not ask for missing student information before generating a roadmap.
    \item BUG-02: Roadmap text is returned as one long paragraph instead of organized sections.
    \item BUG-03: Invalid or empty input causes an unhandled error on the backend.
    \item BUG-04: AI service timeout or failure is not communicated clearly to the user.
\end{itemize}

\section*{TestRail Snapshot 2 Test Plan Summary}

For Snapshot 2, a TestRail test run is created to validate the basic behavior of the AI Career Roadmap Generator.

\subsection*{Key Test Cases}
\begin{itemize}
    \item \textbf{TC-CR1: Roadmap Request with Complete Information} \\
    \textit{Steps:} The student provides a major and career goal and requests a roadmap. \\
    \textit{Expected Result:} A structured multi-phase roadmap is displayed in the chat.

    \item \textbf{TC-CR2: Handling Missing Inputs} \\
    \textit{Steps:} The student requests a roadmap without specifying a major. \\
    \textit{Expected Result:} The chatbot asks a follow-up question to gather the missing major before generating the roadmap.

    \item \textbf{TC-CR3: Invalid Input Text} \\
    \textit{Steps:} The student enters nonsensical text that cannot be mapped to a career goal. \\
    \textit{Expected Result:} The system responds with a clarification prompt asking the student to restate their major or target role.

    \item \textbf{TC-CR4: AI Service Timeout} \\
    \textit{Steps:} A roadmap request is made while the AI service is unavailable or slow. \\
    \textit{Expected Result:} The system returns a friendly error message explaining that the roadmap service is temporarily unavailable.

    \item \textbf{TC-CR5: Response Formatting} \\
    \textit{Steps:} A standard roadmap request is executed. \\
    \textit{Expected Result:} The roadmap is organized into headings or bullet lists so that the guidance is easy to read.
\end{itemize}

These Jira and TestRail summaries document the planning and quality assurance activities attached to Snapshot 2 and can be expanded in future snapshots as the implementation matures.


\end{document}
