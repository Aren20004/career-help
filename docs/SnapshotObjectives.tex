\documentclass[12pt]{article}

\usepackage{geometry}
\geometry{margin=1in}

\title{Snapshot Objectives}
\date{}
\begin{document}
\maketitle

\section*{Snapshot 1 — Start Objectives}

The purpose of Snapshot 1 is to establish the foundation of the Career Help Platform. 
This phase focuses entirely on planning, project structure, and documentation requirements. 
The system architecture, goals, and initial design decisions are defined here.

\begin{itemize}
    \item \textbf{Define the project scope:}  
    The Career Help Platform assists users in selecting career paths and provides curated recommendations for colleges, certificates, internships, jobs, and experience-building opportunities.

    \item \textbf{Establish core architecture:}  
    A web-based frontend, backend recommendation engine, and a structured data layer are planned.  
    A future paid counseling module will be integrated in later snapshots.

    \item \textbf{Create initial documentation:}  
    Version 1.0 of the SRS, SDD, README/User Manual, and Design Specification are created.

    \item \textbf{Plan project dependencies and technologies:}  
    The system design includes Python Flask for backend planning and HTML/CSS/JS for the frontend (no code implementation required at this stage).  
    Docker containerization will be incorporated in future snapshots.

    \item \textbf{Create Jira board:}  
    Tasks are created for UI layout, backend logic planning, data design, documentation work, and future bug tracking.

    \item \textbf{Create workflow diagram:}  
    A high-level depiction showing user input flowing through the UI, backend recommendation logic, data retrieval, and final results displayed to the user.
\end{itemize}


\section*{Snapshot 2 --- First Checkpoint Objectives}

\subsection*{Reflection on Snapshot 1}
Snapshot 1 focused on setting up the foundation of the Career Help Platform. 
The team created the initial versions of the SRS, SDD, User Manual, and Design Specification, 
and defined the overall problem space: helping students explore careers and understand the paths required to reach them. 
At this stage, the chatbot, AI logic, and dynamic roadmap generation were still in the planning phase.

\subsection*{New Objective: AI-Powered Career Roadmap Generator for Students}
Snapshot 2 introduces the first major functional feature of the system: an AI-powered career roadmap generator 
targeted specifically at students. The feature will provide structured multi-step guidance to help students move from 
their current academic status toward their desired career.

\subsection*{Goals for Snapshot 2}
\begin{itemize}
    \item Implement a new backend service that accepts student inputs such as major, interests, experience level, and target career goal.
    \item Integrate an AI model or external API capable of generating a multi-year roadmap with clear milestones.
    \item Extend the chatbot interface so students can request a ``career roadmap'' using natural language.
    \item Include recommendations for relevant classes, internships, projects, and skills that support the student\'s target career.
    \item Ensure responses are formatted in phases (for example, Year 1--4 or Semester 1--8) to help students plan over time.
    \item Lay the groundwork for future extensions such as class schedule lookup and internship search integration.
\end{itemize}

\subsection*{Deliverables for Snapshot 2}
\begin{itemize}
    \item Updated SRS, SDD, User Manual, and Design Specification including the career roadmap generator feature.
    \item Updated workflow diagram showing the new AI roadmap component integrated with the chatbot.
    \item Backend endpoint stubbed or implemented for generating roadmaps.
    \item Basic test cases documented in TestRail for roadmap requests, validation, and error handling.
\end{itemize}


\end{document}
